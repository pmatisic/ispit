%%%%%%%%%%%%%%%%%%%%%%%%%%%%%%%%%%%%%%%%%
% LaTeX predložak za natjecanja u programiranju 
%
% Ovaj predložak je preuzet s:
% http://www.latextemplates.com
%
% Izvorni autor:
% Ted Pavlic (http://www.tedpavlic.com)
% Licenca: CC Attribution-NC-SA 3.0.
%
% Posebno uređeno za opis problema natjecateljskog programiranja
% od koosaga (http://koosaga.com), seungwonpark (http://swpark.me)
%
% GitHub repozitorij za ovaj predložak (trenutno (2017.05.21) privatno):
% https://github.com/seungwonpark/PS-latex-predložak
%
% GitHub repozitorij za ovaj predložak (trenutno (23.02.2024.) privatno):
% https://github.com/pmatisic/ispit
%%%%%%%%%%%%%%%%%%%%%%%%%%%%%%%%%%%%%%%%%

\documentclass{article}
\RequirePackage[croatian]{babel}
\usepackage[utf8]{inputenc}
\usepackage{fancyhdr} % Potrebno za prilagođene zaglavlja
\usepackage{lastpage} % Potrebno za određivanje posljednje stranice za podnožje
\usepackage{extramarks} % Potrebno za zaglavlja i podnožja
\usepackage[usenames,dvipsnames]{color} % Potrebno za prilagođene boje
\usepackage{graphicx} % Potrebno za umetanje slika
\usepackage{listings} % Potrebno za umetanje koda
\usepackage{courier} % Potrebno za font courier
\usepackage{lipsum} % Koristi se za umetanje lažnog teksta 'Lorem ipsum' u predložak
\usepackage{amsthm,amsmath} % Postavljanje jednadžbi
\usepackage{algorithm, algpseudocode} % algoritam
\usepackage{verbatim} % za komentar, okruženje verbatim
\usepackage{spverbatim} % automatski prelom retka okruženje verbatim
\usepackage{listings} % Potrebno za lstlisting
\lstset{breaklines=true} % Prelom riječi unutar okruženja listings
\lstset{basicstyle = \ttfamily,columns=fullflexible}
\usepackage{hyperref} % Potrebno za korištenje href
\usepackage{pgffor} % Potrebno za korištenje \foreach

\lstset{columns=fullflexible}
\DeclareGraphicsExtensions{.pdf,.png,.jpg}
\usepackage{datetime} % Koristi se za prikazivanje verzije kao posljednje vrijeme izmjene
\yyyymmdddate
\usepackage{multicol}
\setlength{\columnseprule}{0.4pt}

% Margine
\topmargin=-0.45in
\evensidemargin=0in
\oddsidemargin=0in
\textwidth=6.5in
\textheight=9.0in
\headsep=0.25in

\linespread{1.1} % Razmak između linija

%%%%%%%%%%%%%%%%%%%%%%%%%%%%%%%%%%%%%%%%%
% Postavljanje zaglavlja i podnožja
\pagestyle{fancy}
\lhead{}
\chead{STEM Igre 2024. - Selekcijski ispit (FOI)} % Srednje zaglavlje na vrhu
\rhead{Subota, 24. veljače 2024.} % Desno zaglavlje na vrhu
\lfoot{\lastxmark} % Donje lijevo podnožje
\cfoot{\thepage} % Donje središnje podnožje
\rfoot{Zadnja izmjena: \today \ \currenttime}
\def\inputdataname{Ulaz } % Naziv 'ulaza'
\def\outputdataname{Izlaz } % Naziv 'izlaza'
%%%%%%%%%%%%%%%%%%%%%%%%%%%%%%%%%%%%%%%%%

\renewcommand\headrulewidth{0.4pt} % Veličina linije zaglavlja
\renewcommand\footrulewidth{0.4pt} % Veličina linije podnožja

\setlength\parindent{0pt} % Uklanja sve uvlačenje iz odlomaka

\setcounter{secnumdepth}{0} % Uklanja zadane brojeve odjeljaka
\newcounter{homeworkProblemCounter} % Stvara brojač za praćenje broja problema

\newcommand{\iodataNo}[1]{%
\begin{minipage}{\textwidth}
\begin{multicols}{2}
\href{run:input#1.txt}{\inputdataname#1} \\
\rule{\columnwidth}{1pt}
\lstinputlisting[language={}]{input/input#1.txt}
\columnbreak
\href{run:output#1.txt}{\outputdataname#1} \\
\rule{\columnwidth}{1pt}
\lstinputlisting[language={}]{output/output#1.txt}
\end{multicols}
\vspace{\baselineskip}
\end{minipage}
}

\begin{document}

\section{Problem 1. Permutacije}

\begin{center}
Vremensko ograničenje po testu: 1 sekunda \\
Ograničenje memorije po testu: 256 megabajta \\
Ulaz: standardni ulaz \\
Izlaz: standardni izlaz \\
\end{center}
\textit{Za svoj rođendan, Ivica je dobio permutaciju!} \\
Permutacija je niz cijelih brojeva duljine $n$, tako da se svi cijeli brojevi od 1 do $n$ pojavljuju točno jednom. Ivica može u jednom potezu zamijeniti bilo koja dva susjedna broja. Sada je Ivici zanimljivo može li sortirati svoju permutaciju u parnom broju poteza (napraviti paran broj zamjena susjednih brojeva tako da postigne permutaciju $1, \ldots, n$). Ivica je još uvijek mali, pa je zatražio tvoju pomoć!

\subsection{Ulaz}
Prvi redak sadrži cijeli broj $n$ ($1 \leq n \leq 100$) - duljinu permutacije. \newline

Drugi redak sadrži $n$ cijelih brojeva $p_1, \ldots, p_n$ koji predstavljaju permutaciju.

\subsection{Izlaz}
U jednom retku ispiši "Yes" ili "No" (bez navodnika).

\subsection{Primjeri}
% Stavite tekstualne datoteke ulaza u direktorij 'input/'.
% Format: 'input/input%d.txt'
% Stavite tekstualne datoteke izlaza u direktorij 'output/'.
% Format: 'output/output%d.txt'

\iodataNo{1}
\iodataNo{2}

% % Ili možete koristiti petlju for...
% \foreach \i in {1, 2, ..., 10}{
% \iodataNo{\i}
% }

\subsection{Napomena}
Napravite svoje testne slučajeve (ulaze) kako biste testirali ispravnost vašega rješenja, s obzirom da validacijski sustav osim navedenih testnih slučajeva ima i bazu ostalih različitih testnih slučajeva prema kojima će se testirati vaše rješenje. Također imajte na umu da vaše rješenje ne prelazi vremensko i memorijsko ograničenje po testu jer se i ona uzimaju u obzir bodovanja.

\newpage

\section{Problem 2. Upravljanje zgradom}

\textit{Ivica je odrastao i postao biznismen!} \\
Otkako je Ivica ušao u poslovne vode, napravio je zgradu u Varaždinu i njegov cilj je sada osigurati lakše upravljanje tom zgradom tako što počinje s njenom digitalizacijom. Njegova ideja je dizajnirati sofisticirani sustav za upravljanje zgradom koji integrira razne funkcije poput kontrole klime, sigurnosnih sustava, upravljanja energijom, održavanja i nadzora prostora. Sustav treba omogućiti centralizirano upravljanje, automatizaciju i nadzor različitih aspekata zgrade, pružajući prilagođena sučelja za različite korisnike (upravitelj, stanari, održavatelji). Kako bi Ivica ostvario takav sustav, zamolio te je da mu pomogneš sa svojim idejama!

\subsection{Kriteriji dizajniranja rješenja}

\begin{enumerate}
    \item \textbf{Rješenje sustava (5 bodova):}
    \begin{itemize}
        \item komponente sustava i njeni opisi (npr. aplikacija, serveri, podatci, pogledi i sl.)
        \item dijagram arhitekture sustava koji prikazuje međusobne veze i komunikaciju između komponenti
    \end{itemize}
    
    \item \textbf{Aplikacija (5 bodova):}
    \begin{itemize}
        \item način rada aplikacije (npr. frontend, backend, mobilna/desktop/web itd.)
        \item funkcionalnosti aplikacije
    \end{itemize}
    
    \item \textbf{Baza/skladište podataka (5 bodova):}
    \begin{itemize}
        \item odabir vrste i strukture baze/skladišta podataka
        \item obrazloženje odabira takve baze podataka i njenog plana implementacije
    \end{itemize}
    
    \item \textbf{Tehnologija za razvoj (5 bodova):}
    \begin{itemize}
        \item popis tehnologija, programskih jezika, programskih okvira i alata potrebnih za izradu sustava
        \item obrazloženje zašto su odabrane baš te tehnologije i kako one doprinose funkcionalnosti i efikasnosti sustava
    \end{itemize}
\end{enumerate}

\subsection{Napomena}
Navedeni kriteriji su okvirni plan bodovanja. Nije potrebno opširno pisati odgovore, već u rečenicu-dvije opisati jasnu i preciznu namjeru rješenja kod konkretne stavke kriterija. Naravno, ukoliko imate svoje dodatne inovacije, obrazloženja, opise, dijagrame i slično, to će se također uzeti u obzir. Pritom se napominje da rješenje sustava treba biti realno za izvođenje. Rješenja (opise, slike dijagrama i ostalo) možete pisati u bilokojem tekst \textit{editor}-u, ali pri predaji je potrebno pretvoriti ga u \textbf{pdf} i takvog predati.

\end{document}